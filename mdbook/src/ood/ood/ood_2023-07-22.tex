

\graphicspath{{./img/}}

\section{Maintaineable Code With SOLID Concept}

\subsection{Single Responsibilities Principle}
pecah fungsi ke beberapa method, jangan buat seluruh fungsi dalam 1 method





\subsection{Open Close Principle}
gunakan interface untuk membuat object yang memiliki method yang sama. 

\subsection{Liskov Subtitution Principle}
menurut barbara liskov pada presentasinya di konferensi \textit{Data Abstraction and Hierarchy}
setiap object dari subtype harus \textit{suitable} dan dapat bekerja denga superclassnya
hal ini disebutnya sebagai \textit{Liskov Subtitution Principle}. 
salah satu cara untuk menghindari \textit{Liskov Subtitution Principle} atau LSP
adalah dengan menggunakan prinsip \textit{Interface} dibandingkan menggunakan \textit{Inheritance}

\subsection{The Interface Segregation Principle}
dalam ISP atau interface segregation Principle, jangan gunakan method atau variable yang tidak diperlukan dalam class, 
dalam C\# dapat digunakan \textit{Inheritance of Interface}.

\begin{lstlisting}

  public interface ITwoDeeShape {
    public double Width{get; set;}
    public double Height{get; set;}
  }

  public class SquareISP : ITwoDeeShape{
    // implement all interface 
  }

  public interface IThreeDeeShape : ITwoDeeShape{
    // add new properties to our new interface
    public double Depth{get; set;}
  }

\end{lstlisting}


\subsection{Dependencies Inversion Principle}
\begin{itemize}
  \item High-level classes or modules should not depend on low-level classes or modules and both should depend on abstractions.
  \item Abstractions should not depend on implementation details but instead details should depend upon abstractions.
\end{itemize}


\section{Getting Creative With Creational Pattern}

\subsection{Factory Method Pattern}





