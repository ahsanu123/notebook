\raphicspath{{./img/}}

\begin{document}

%=========================
% Add A logo
%=========================

\section{MOSFET [rashid]}

mosfet adalah device unipolar, dengan arus bergantung pada majority carrier. arus mosfet dapat di kendalikan
dengan menggunakan medan elektrik. MOSFET dibagi menjadi 2 mode Enhancement dan Depletion mode. setiap mode mosfet dibagi 
2 tipe, yaitu NMOS dan PMOS.

\Figure[width=10cm, caption={fuck you MOSFET}, placement=h]{galirMosfet}

\subsubsection{NMOS Enhancement}
\Figure[width=14cm, caption={NMOS Structure and Symbol}, placement=h]{nmosStructure}
simbol NMOS terlihat pada gambar 2.c, simbol tersebut dapat di singkat menjadi gambar 2.d [abbreviated] (!arah panah yang berkebalikan)
NMOS bekerja dengan tegangan postif.

\subsubsection{PMOS Enhancement}
\Figure[width=14cm, caption={PMOS Structure and Symbol}, placement=h]{pmosStructure}
simbol PMOS terlihat pada gambar 3.c, simbol tersebut dapat disingkat atau disederhanakan seperti gambar 3.d. 
PMOS bekerja dengan tegangan negatif

\subsection{Operation Mode}
terdapat 3 buah mode operasi pada mosfet, yaitu cutoff region, ohmic region (terbagi menjadi 2 yaitu linear dan nonlinear),
dan saturation region. cutoff region terjadi ketika $0 <= VGS < VT$, dimode ini mosfet seperti 2 buah dioda yang terpasang 
berkebalikan. 

region operation ke-2 yaitu Linear region, pada buku [rashid] linear mode dibagi menjadi 2 yaitu 
linear ohmic region dan nonlinear ohmic region,\textbf{linear ohmic region terjadi ketika. } 

\begin{gather}
  VGS > VT  \\
  0 < VGS << (VGS-VT)  
\end{gather}

ketika $VGS > VT $ namun nilai tegangan $0< VDS< (VGS - VT)$, mosfet berada dalam kondisi nonlinear
ohmic region.

region operation ke 3 yaitu \textbf{saturation region}, saturaion region terjadi ketika $VGS > VT$ dan 
$VDS > (VGS - VT)$. berikut bias menggunakan tegangan DC.


\Figure[width=12cm, caption={DC bias MOSFET}, placement=h]{dcBiasPmosNmos}


\subsubsection{Cut Off Region}

\subsubsection{Linear Ohmic Region}
pada kondisi linear ohmic arus $i_D$ dapat dihitung menggunakan hukum ohm $i_D = V_{DS}/r_{DS}$. 
konduktansi antara drain dan source dapat dihitung menggunakan.

\begin{gather}
  g_{DS} = \frac{1}{r_{DS}} = \frac{W}{L} \mu_n Q_n
\end{gather}
 
dimana :\\
$g_{DS}$ = konduktansi channel\\
$\mu_n$ = mobilitas elektron in reverse bias dalam oxida\\
$Q_n$ = magnitude reverse bias layer\\
$W$ = channel Width\\
$L$ = channel length\\


\subsubsection{Nonlinear Ohmic Region}
pada nonlinear ohmic region $i_D$ dapat dihitung menggunakan persamaan dibawah ini

\begin{gather}
  i_D = \frac{K_m}{2} (2(VGS-VT)VDS - VDS^2)
\end{gather}
dimana: \\
$K_m$ = $\frac{W}{L}$konstanta fabrikasi transistor.\\
$K_n$ = $K_m/2$\\

\Figure[width=8cm, caption={nonlinear $i_D$ vs $VDS$}, placement=h]{nonlinearOhmicRegion}

rumus diatas dapat digunakan jika kondisi transistor memenuhi persamaan dibawah ini.

\begin{gather}
  VGS > VT \\
  VDS < (VGS-VT)
\end{gather}

\subsubsection{Saturation Region}
operation region yang terakhir adalah saturation region. pada daerah ini kenaikan $i_D$ tidak lagi signifikan
(mengalami \textit{saturation}). dengan menurunkan persamaan nonlinear ohmic region dan men-samadengankan dengan 0
didapatkan persamaan $i_D$ pada mode saturation region sebagai berikut. 

\begin{gather}
  \frac{di_D}{dV_{DS}} = \frac{K_n d}{dV_{DS}}[2 (v_{DS} - V_t) V_{DS} - V_{DS}^2] = 0 \\
  i_D = K_n (VGS - VT)^2
\end{gather}
diamana:\\
$K_n$ = $K_m/2$ = $W/L2$ = konstanta fabrikasi mosfet \\
$VT$ = thershold voltage = tegangan threshold \\ 


\subsubsection{Output dan Transfer karakteristik}
ketika berada dalam kondisi cutoff, drain dan source mosfet seperti dua buah dioda yang dipasang berkebalikan.


ketika berada dalam linear region mosfet seperti sebuah resistor yang hambatannya diatur oleh tegangan gate.
$VGS > VT$ dan $VDS < (VGS-VT)$
\begin{gather} 
  i_D = \frac{K_m}{2} (2(VGS-VT)VDS - VDS^2)
\end{gather}

ketika berada dalam saturation region. $VGS > VT$ dan $VDS > (VGS-VT)$
\begin{gather}
  i_D = K_n (VGS-VT)^2
\end{gather}

\end{document}

